\section{Zooming}
The zoom feature for the image annotation project was based upon those found in Internet browser software such as FireFox and Safari, and it comprises three main functions: zoom in, zoom out, and reset zoom.  Since browsers are arguably the most frequently used software on a computer it seemed natural to follow their implementations of this feature.  This included using the same shortcut keys as those found in browser software.  As a partially sighted person myself I naturally hit these key combinations first whenever I want to zoom in or out of an object on the screen, and anecdotal evidence seems to suggest that this has become the de facto standard; whenever an application doesn’t follow this convention I find it initially confusing and then increasingly irritating as I hunt around the menu structure to see if this feature is even implemented.  I consider this very poor design practice and this is why I chose to implement the zoom feature in this way.

In terms of criticisms, the zoom feature was designed to scale the entire image up and down at the user’s discretion.  This is contrary to some other image editing software packages which chose to implement this feature in a different way.  The alternative method is to have a single function that, depending upon where you click, will zoom and center on a particular location on the left-click of a mouse, and zoom out upon a right-click.  This can be useful when the user wishes to edit an image in fine detail; however, the primary use case of this product seems to suggest that annotations will not require fine detail creation and hence this is why it was implemented in this way.  Still, this is a design decision and should be considered a legitimate concern.  If this were a commercial product usability testing would be useful to bore out which method is more conducive to the programs users.