\section{Save\slash Load}
Another feature of this package is saving annotations of an image and capability of loading them into the editor later. To do this program saves location of every vertex and the name of each label in an XML file and store it in the same directory where the image is located. Once this file is saved, user will see the result next time image is opened. In 

Choice of XML format for saving annotations is based on two reasons. First, we needed a standard method of storing the meta-data and the widely used standard is XML. We see that most of the interoperable file formats make use of XML one way or another. The other reason was to make the stored information human readable so when the file is opened in a text editor user can read and edit the annotations.

Program loads the meta-data of every image automatically if the appropriate XML file exists in the same directory under the name of that image. This scheme of storing was chose to prevent making Data Bases of labels. This has some advantages disadvantage. The advantages are that first, there is no need to fetch and alter labels of an specific image in the database(Depending on scheme of the database, it can be computationally expensive if huge number of images are considered). Also, this method keeps security of meta-data at the same level of images so when a user does not have access to a image he will not be able to read annotations too and if he has only certain level of access, for example read-only, then he will be able to read annotations too. The drawback of this method of keeping information is that it is not easy to implement other features such as search over image labels.Since the use case of the package does not cover search, this method of storing information is chosen.

An example produced file can be found in listing \ref{lst:svd}.    

\begin{center}
\lstset{language=XML, basicstyle=\footnotesize\ttfamily, caption=XML file used to store image labels,frame=single,captionpos=b,label=lst:svd}
\lstinputlisting{svd.txt}
\end{center}