\section{Fast Browsing}
The fast browsing capabilities of this product consist of two docks.  One holds the hierarchical file structure of the system and the other is a thumbnail viewer.  The thumbnail viewer dock is directly linked in functionality to the file viewer dock in that it displays the images found in the deepest expanded folder in the file system tree.  Both docks can be used to open an image file by double clicking on the respective icon or image.  This allows a user to open image files more quickly without the use of the open dialog box in the file menu, and hences makes it easier to annotate multiple images at once, or indeed browse to find the correct image the user was looking for if they are unsure of the file name.

These docks have been deliberately placed at the left side of the application by default, although they are able to be individually dragged to the right side if the user so chooses or simply allowed to float at a desired location.  The reason these features were placed in this way is based upon the observation that navigation controls typically reside to left of the screen.  From the Windows Explorer application in Microsoft Windows 95, to the Facebook website, to the integrated development environment used to construct this product, navigation controls more often than not are placed to the left hand side of the screen.  This has become the unwritten convention and would be foolish to ignore without good cause.

In evaluating the implementation of the fast browsing features there are two main negative comments one could make.  The first relates to the fact that an action on one dock, expanding or collapsing a folder, has a direct impact upon another dock, the thumbnails change.  Due to that connection a person may argue that these controls should be present on the same dock to avoid confusion.  By having these two controls on top of each other in a single dock it may be true to say that it would be clearer that they inter-operate with each other; however, I would argue that this ignores the cases where a user wishes to only navigate through the folder structure and doesn’t want a thumbnail viewer shown on their workspace - or vice-versa.  The other criticism is that there is no mechanism, through a shortcut key or otherwise, for a user to skip from one image to the next within a single folder when batch annotating multiple images without using the mouse.  In defense of this fact I would argue that since the act of annotating images requires heavy use of the mouse, providing keyboard navigation shortcuts would not make batch image annotation any faster.
