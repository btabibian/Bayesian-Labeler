\section{Image Annotation}
Image annotation in this program uses a set of vertices which form a polygon to distinguish the object from the rest of the image. User, once clicked on "Add" button, can select any where on the screen and start drawing the polygon. When finished, right click on the picture or click on "Add" button again finishes drawing.Then, A window will pop up and asks for a name for annotated part of the picture, This window is modal window provided by the framework and is an standard way of getting small inputs. Names appear as a tool-tip of each annotation.

There are several remarks about this feature. One problem that we had to deal with was that users may not select the vertices of the polygon in order. This situations is not supported by the underlying framework and produces undefined shapes. Our solution to this problem was to colour the surface of the polygon as user adds vertices. This results to an intuition for the user which prevents him to add a vertex Between to vertex of the polygon.

We also dropped editing position of each vertex of a polygon. Therefore, once user adds an annotation he can not change it unless annotation is deleted and redrawn. This design decision was based on time constraints that we had and other features that we wanted to implement. Also , in practice, annotations are usually small and therefore user prefers to redraw the object rather than changing and/or adding extra vertices to an already drawn polygon.