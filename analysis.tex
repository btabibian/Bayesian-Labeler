\section{Menu and Shortcuts}

Overall the participant in our tests found the program to be easy to use after a few tries, but there were a few stumbling blocks they encountered whilst using the program.  Namely these related to the placement of user interface components and the implementation of the canvas.

It took the user 56 seconds in total to load an image file, and most of that time was spent hunting the interface for the open command.  The creators of the application labelled their button load, which is arguably a less common term than open, but the main problem that the user encountered was that the vast majority of desktop applications they have used have a main menu with a file sub-menu and an open menu item - this program does not.  Also, because the action buttons seem fairly arbitrarily placed around the canvas, this will necessarily add to the confusion of first-time users.

The other task that the user found particularly difficult was the selecting of an annotation.  This took them 45 seconds to complete, and in terms of poor interface design there were two main issues.  The first thing that the user tried to do was select the edge of the polygon to highlight it.  This is a fairly natural action since it is usually the case that a UI object can be selected by clicking on it; however, in this case the canvas interpreted the click incorrectly as the users wish to create another polygon.  This is poor interface design since the designers of the program should have imagined that the user would try to select an annotation in this way and either implement it or communicate to the user that the edges of the polygon are not click-able: perhaps by using a broken line as opposed to a continuous one.  After the user recovered from their error they realised that the selection of an annotation must be accomplished somewhere else on the window and this illustrates the second issue with this feature.  It took some time for the participant of the test to realise there was a list of their annotations to the far right of the window.  User interface components are typically placed to the top and the left of the screen so by placing a piece of information on the far right it is less likely that a persons eye will travel that way as often.  This is fairly well known in interface design, and so the creators of this product should have given more thought to the placement and implementation of this feature.
